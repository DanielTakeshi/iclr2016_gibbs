\documentclass{article} % For LaTeX2e
\usepackage{iclr2016_conference,times}
\usepackage{hyperref,algorithm,algpseudocode,array,tabularx,multirow,caption,subcaption,amsfonts,url,verbatim,enumitem,amsmath,graphicx}
\usepackage{url}

\title{Fast Parallel SAME Gibbs Sampling on \\ General Discrete Bayesian Networks}

\author{Daniel Seita, Haoyu Chen \& John Canny \\
Computer Science Division \\
University of California, Berkeley \\
Berkeley, CS 94720, USA \\
\texttt{\{seita,haoyuchen,canny\}@berkeley.edu}
}

% The \author macro works with any number of authors. There are two commands used to separate the
% names and addresses of multiple authors: \And and \AND.
%
% Using \And between authors leaves it to \LaTeX{} to determine where to break the lines. Using \AND
% forces a linebreak at that point. So, if \LaTeX{} puts 3 of 4 authors names on the first line, and
% the last on the second line, try using \AND instead of \And before the third author name.
%
% ICLR requires electronic submissions, processed by \url{http://arxiv.org}. See ICLR's website for
% more instructions.
% 
% If your paper is ultimately accepted, the statement {\tt {\textbackslash}iclrfinalcopy} should be
% inserted to adjust the format to the camera ready requirements.
% 
% The format for the submissions is a variant of the NIPS format.  Please read carefully the
% instructions below, and follow them faithfully.

\newcommand{\fix}{\marginpar{FIX}}
\newcommand{\new}{\marginpar{NEW}}

%\iclrfinalcopy % Uncomment for camera-ready version

\begin{document}

\maketitle

\begin{abstract}
A fundamental task in machine learning and related fields is to perform inference on Bayesian
networks. Since exact inference takes exponential time, it is common to use an approximate algorithm
such as Gibbs sampling, but this can still be intractable for graphical models with just a few
hundred binary random variables. In this paper, we address this issue by presenting our highly
optimized Gibbs sampler, which we believe is the fastest one publicly available.  Our Gibbs
sampler is GPU-accelerated, heavily parallelized, and replicates data via State Augmented Marginal
Estimation (SAME) to decrease convergence time while reaching higher quality parameter estimates.
Experiments on both synthetic and real data show that our Gibbs sampler is multiple orders of
magnitude faster than the state of the art sampler, JAGS, without sacrificing accuracy. Our ultimate
objective is to introduce the Gibbs sampler to researchers in many fields to expand their range of
feasible inference problems.
\end{abstract}




\section{Introduction}\label{sec:intro}

In many machine learning applications, one has a distribution $P(X,Z \mid \Theta)$ where $X$ is
observed data, $Z$ is hidden (latent) data, and $\Theta$ represents the model parameters. The goal is
generally to find an optimal $\Theta$ with respect to $X$, while marginalizing out $Z$. To represent
these problems, it is common to use graphical models, which combine probability theory and graph
theory to present a robust formalism for probabilistic inference. A Bayesian network is a graphical
model defined by a directed acyclic graph and a set of conditional probability tables (CPTs). Each
CPT represents a local probability distribution $\Pr(X_i \mid X_{\pi_i})$ where $X_i$ is a random
variable, and $X_{\pi_i}$ represents its set of parent nodes in the graph. We denote the full set of
CPTs as $\Theta$.

In this paper, our focus is on parameter estimation of Bayesian networks with discrete random
variables (``discrete Bayesian networks'') based on partially observed data $\mathcal{D} = \{\xi_1,
\ldots, \xi_m\}$, where $\xi_i$ is an $n$-dimensional vector with assignments to the $n$ variables
of the graph, or ``N/A'' to indicate missing data. We assume that the structure of the Bayesian
network --- its nodes and edges --- is known in advance. This type of problem often arises in
practice because it is easier to elicit graph structure from human experts than it is to get
numerical parameters~\citep{Koller2009}. Furthermore, it is often unrealistic to expect our data
$\mathcal{D}$ to be completely observed, as data might be missing due to errors (e.g., human
oversights) or deliberate omissions.

Well-known strategies for parameter estimation with partially observed data include
Expectation-Maximization~\citep{EMpaper} and variations of gradient ascent~\citep{Thiesson95}.
Parameter estimation using these methods requires running probabilistic inference over the missing
data, which tends to be the limiting factor since  exact inference using the junction tree algorithm
takes exponential time. It is therefore common to use approximate inference procedures.  One way is
to perform Markov Chain Monte Carlo (MCMC) simulation, where one constructs a Markov chain whose
states are an assignment to all unobserved variables, such that the stationary distribution of the
chain is the posterior probability over these variables.

Gibbs sampling~\citep{Geman1984} is a special case of MCMC simulation, which at iteration $t$, goes
through each unobserved variable and samples from its full conditional based on the samples from the
current or previous iteration: $X_i^{(t)}\ {\raise.17ex\hbox{$\scriptstyle\sim$}} \
\Pr(X_i^{(t)} \mid X_1^{(t)}, \ldots, X_{i-1}^{(t)}, X_{i+1}^{(t-1)}, \ldots, X_n^{(t-1)})$. For the
parameter estimation problem, one also needs to sample to update $\Theta$ from $P(\Theta
\mid \mathcal{D})$, the posterior distribution over the complete data data $\mathcal{D}$
(``complete'' due to sampling). Since we assume discrete random variables, the samples are
multinomial counts, so for Bayesian estimation, one can impose a set of Dirichlet priors on
$\Theta$ so that the posterior is also a set of Dirichlets.

While Gibbs sampling is commonly used in machine learning, it is a very broad technique that may not
be able to match the performance of special-purpose inference algorithms without extensive
fine-tuning or using complicated variations~\citep{Murphy2012}. \textbf{Daniel: I am not sure if I
am explaining this last sentence well enough. Murphy's book has some explanation, but not much.} In
a recent result,~\citet{SAME2015} showed that by combining the State Augmented Monte Carlo (SAME)
technique~\citep{SAME2002} with Gibbs sampling, one can get fast, high quality parameter estimates
for discrete graphical models, but they only applied it to two specific models that do not have
mutual dependencies between discrete states.  In this paper, we build upon that result by
presenting a SAME Gibbs sampler for general discrete Bayesian networks. To be precise, the
novelty and aspects of our Gibbs sampler is that

\begin{itemize}[noitemsep]
    \item our sampler uses an adjustable SAME parameter to replicate data, causing the Gibbs sampler
    to converge faster to higher-quality MAP or ML parameter estimates. We believe this is due to
    reduction of excess variance from standard Gibbs sampling.
    \item we run our sampler on state of the art GPUs and parallelize it as much as possible. Our
    sampler can ultimately scale up to problems with hundreds of random variables.
    \item the sampler is designed to maximize throughput on only one computer, thus avoiding the
    need to work though complicated distributed systems.
    \item it is open source as part of the BIDMach
    library\footnote{\url{https://github.com/BIDData/BIDMach}} and comes with various diagnostic
    tools. Furthermore, BIDMach's mini-batch updating and matrix caching features mean we could even
    run our Gibbs sampler on data too large to fit in a computer's RAM.
\end{itemize}

We benchmark our sampler versus the state of the art Gibbs sampler, JAGS~\citep{JAGS2003}, and show
that our Gibbs sampler is multiple orders of magnitude faster. Consequently, in addition to
introducing our Gibbs sampler to researchers, we argue in this paper that Gibbs sampling augmented
with SAME can be competitive with or superior to the fastest special purpose inference algorithms
for Bayesian inference. \textbf{Daniel: my main concern is that we are not talking about these
``special purpose inference algorithms.'' We're just presenting a fast SAME Gibbs sampler.  Also, I
don't know a good way to incorporate factor graphs here; if I did, then we can
cite~\citep{Factorie2009}.}





\section{Related Work}\label{sec:related_work}

The problem of Bayesian inference for graphical models is old and well-studied, and for reference,
we refer the reader to textbooks (e.g., see~\citet{Koller2009} for a broad background) and survey
papers (e.g., see~\citet{Wainwright2008} for a discussion on variational inference as an alternative
to MCMC methods). \textbf{Daniel: I would like to have a survey paper that actually compares Gibbs
sampling versus other methods. I think this paragraph should be expanded, while condensing the
introductory section.}

Gibbs sampling is also relatively old and well-studied, but it has recently been getting more
attention as the research community explores efforts to improve speed and scalability. The Gibbs
sampling research most related to our contribution involves efforts to (1) parallelize Gibbs
sampling and (2) achieve high throughput. By exploiting the conditional independence assumptions of
Bayesian networks, is possible to have an exact, semi-parallel Gibbs sampler that iterates through
color groups of nodes, and within each group, sampling each node independently~\citep{Gonzalez2011},
a strategy that we employ in our sampler. It is also possible to relax the sequential nature of the
algorithm and approximate Gibbs sampling by limiting global communication~\citep{Johnson2013}.  The
databases community has also been able to contribute to speeding up Gibbs sampling by showing how
improved system design can lead to higher throughput~\citep{Zhang2013}.

Recently, the use of Graphics Processing Units (GPUs) has become essential for developing high
performance software, as exemplified by popular packages such as Theano for evaluating mathematical
expressions with multidimensional arrays~\citep{Theano2012} and CAFFE for neural
networks~\citep{jia2014caffe}.  The Augur probabilistic programming language~\citep{Augur2014},
which generates inference code for Bayesian networks, demonstrates the importance of combining GPU
code with extra parallelism introduced from exploiting conditional independence assumptions.

The result most directly related to our paper, as briefly mentioned in Section~\ref{sec:intro}, is
one that shows how the addition of SAME to a GPU-accelerated Gibbs sampler can be very fast for
Latent Dirichlet Allocation and the Chinese Restaurant Process~\citep{SAME2015}. In that paper, they
explored the application of SAME to graphical model inference on modern hardware, and showed that
combining SAME with factored sample representation (or approximation) gives throughput competitive
with the fastest symbolic methods, but with potentially better quality. We (non-trivially) extend
this result by implementing a more general-purpose Gibbs sampler that can be applied to arbitrary
discrete graphical models.





\section{Fast Parallel SAME Gibbs Sampling}\label{sec:same}

SAME is a variant of MCMC where one artificially replicates data to create distributions that
concentrate themselves on the global modes~\citep{SAME2002}. It is an efficient way of performing
MAP estimation in high-dimensional spaces when needing to integrate out a large number of variables.
Given a distribution $P(X,Z\mid \Theta)$, to estimate the most likely $\Theta$ based on the data
$(X,Z)$ using SAME, one would define a new joint $Q$:
\begin{equation}\label{eq:same}
Q(X,\Theta,Z^{(1)},\ldots,Z^{(m)}) = \prod_{j=1}^m P(X,\Theta,Z^{(j)})
\end{equation}
which models $m$ copies of the distribution tied to the same set of parameters $\Theta$, which in
our case forms the set of Bayesian network CPTs. This new distribution $Q$ is proportional to a
\emph{power} of the original distribution, so $Q(\Theta \mid X) \propto (P(\Theta \mid X))^m$. Thus,
it has the same optima, including the global optimum, but its peaks are sharpened~\citep{SAME2002}.
Note that as $m$ increases, SAME approaches Expectation-Maximization~\citep{EMpaper} since the
distribution would peak at the value corresponding to the maximum likelihood estimate.

We argue that SAME is beneficial for Gibbs sampling because it helps to reduce excess variance.  It
is important, however, not to set the SAME replication factor $m$ too high, because that might
result in getting trapped in local maxima in the sharpened distribution. An ideal setup might be to
start with a low replication factor and gradually increase it, a technique similar to simulated
annealing for gradient descent, because both involve ``cooling'' the target distribution to sharpen
the peaks.

\textbf{Daniel: I am really confused about what I should include for this section. I feel like this
just repeats a lot of the SAME description. It might be necessary to do that, but do we have an idea
on what we want to discuss?}




\section{Implementation of SAME Gibbs Sampling}\label{sec:implementation}

\begin{figure}[t]
\centering
\includegraphics[width=0.8\textwidth]{fig_BIDMach_final}
\caption{This visualizes our Gibbs sampler at work. The original data is split into four
mini-batches of equal sizes (for caching purposes), with each having some known data (shaded gray)
and unknown data (white). For each batch, the sampler replicates the data with SAME $m=3$, samples
the unknown values, then uses those to update the CPTs. The resulting samples are stored in memory
and are used as the starting point for when we sample this batch again.}
\label{fig:BIDMach}
\end{figure}

Our Gibbs sampler is implemented and integrated as part of the open-source BIDMach
library~\citep{bidmach} for machine learning.  Figure~\ref{fig:BIDMach} shows a visualization of how
it works on real data. Our sampler expects a (usually sparse) data matrix, with rows representing
variables and columns representing cases. BIDMach divides data into same-sized ``mini-batches'' and
iterates through them to update parameters. Going through all mini-batches is one full pass over the
data.

Our main contribution is introducing a Gibbs sampler augmented with SAME sampling.  Consequently,
if $m$ is the SAME parameter, for each mini-batch our sampler forms $m$ copies of the known data.
Then, it performs Gibbs sampling to fill in the unknown values in each copy of the mini-batch using
the same set of CPTs.  These sampled results are combined with an adjustable Dirichlet prior and (if
desired) the current CPT\footnote{This would be useful if one wanted to do a moving average update
of the CPT. Also, the use of the Dirichlet prior tends to be more important when the data is sparse,
since it ``smooths'' the CPTs.} to form a set of discrete counts that we then sample from to
estimate the updated CPT. To preserve samples across all runs, each time the sampler processes a
mini-batch, it stores the sampled results in memory.  Then, during the next \emph{pass} over the
data (i.e., after having gone through all the batches) it starts the Gibbs sampling process from
that stored data. It is possible to skip the first few samples (the standard \emph{burn-in} period)
and also to only update the parameters every $N^{\rm th}$ iteration to reduce correlated samples.

There are many ways in which we optimize the sampler to maximize throughput on one computer and to
avoid excess memory allocation. First, our sampler is GPU accelerated and takes advantage of
parallelism in modern hardware by implementing the sampling process using matrix operations. Since
memory is scarce on GPUs (3-12 GB is typical), we use a matrix caching strategy to reuse memory for
matrices of the same dimensions. This is why BIDMach works with same-sized mini-batches, since the
batches themselves, along with all other matrices used as part of the sampling computation, are of
fixed size and therefore BIDMach can re-use their memory slots\footnote{This does not generally
apply to the very last mini-batch, but that last batch would not cost too much in extra memory, and
one can even ignore it if needed.}. Even with caching, default mini-batch sizes might be too large,
but since BIDMach is highly customizable, one can reduce the number of instances for each batch
(i.e., columns in the data matrix) to reduce the memory footprint. In fact, this step is often
necessary to use a very high SAME parameter.

As mentioned in Section~\ref{sec:related_work}, we further parallelize Gibbs sampling in an
exact manner via chromatic sampling. In Bayesian networks, we konw that $u, v \in \mathcal{V}$ are
independent conditioning on a set of variables $\mathcal{C}$ if $\mathcal{C}$ includes at least one
variable on every path connecting $u$ and $v$ in $\tilde{\mathcal{G}}$, which is the \emph{moralized
graph} of the network (i.e., the graph formed by connecting parents and dropping edge orientations).
That is, vertex set $\mathcal{C}$ separates the dependency between $u$ and $v$.

Suppose there is a $k$-coloring of $\tilde{\mathcal{G}}$ such that each vertex is assigned one of $k$
colors and adjacent vertices have different colors. Denote $\mathcal{V}_c$ the set of variables
assigned color $c$ where $1 \leq c \leq k$. One can sample sequentially from $\mathcal{V}_1$ to
$\mathcal{V}_k$, where within each color group, it samples all the variables in parallel. This
parallel sampler corresponds exactly to the execution of a sequential scan Gibbs sampler for some
permutation over the variables and will converge to the desired distribution because variables
within one color group are independent to each other given all the other variables. Finding the
optimal coloring of a general graph is NP-complete, but efficient heuristics for balanced graph
coloring perform well in many problems.

\textbf{Daniel: TODO Ahhh, I forgot! The stuff in Figure 1 does actually require data to fit in
memory. There are ways we can get around this with more efficient storage but we should make that
explicit, and say that Figure 1 is only for intuition as most data will fit in memory anyway!}




\section{Experiments}\label{sec:experiments}

We benchmark our Gibbs sampler on one synthetic and one real dataset. We compare it with
JAGS~\citep{JAGS2003}, which is the most popular and efficient tool for Bayesian inference, and also
uses Gibbs sampling as the primary inference algorithm. There are alternative software packages that
could be used for parameter estimation, such as Bayes Net Toolbox~\citep{bnt2001}, but we could not
find one that substantially outperformed JAGS.

We evaluate our sampler and JAGS on a PC equipped with a single 4-core CPU (Intel Core i7-3667U)
and a dual-core GPU (Nvidia GTX-690). Only one core of GPU is used in the benchmark. We also use
Intel VTune Amplifier to profile each program and measure the flops performance. VTune allows users
to attach any running process and monitor all the hardware instructions executed, including both X87
legacy floating point operations and SSE operations.

\textbf{Daniel: I need to fix the preceding paragraph to have the specs for Bitter. I know the GPU
is a GeForce GTX Titan X, but what about the processor, and the number of cores, and anything else?
Also, should we mention the gflops?}

We emphasize that the use of one computer is deliberate and a strength of our sampler, as it is
cheaper and simpler to run as compared to a computing cluster.

\textbf{Daniel: some notes. Integrate this into the writeup: Bitter is quite a bit newer, but its
processor is workstation rather than server grade and has half the cores. For stout: 64 GB of CPU
RAM (according to the free -m command) CPU: Intel Core Xeon, eight cores, 2.20 GHz clock speed
(E5-2650 Sandy Bridge) GPU: NVIDIA GTX 690 (2GB Ram per GPU).  For bitter: 16 GB of CPU RAM CPU:
Intel Core i7, four cores, 3.40 GHz clock speed GPU: NVIDIA GTX Titan X (has about 12 GB of RAM)}

\subsection{Synthetically Generated ``Koller'' Data}\label{ssec:koller_data}

\textbf{Daniel: change all figures to have (Koller) at the end! Also, refer to this as the Koller
data! I.e., figures should have the title: XXX (Koller) or XXX (MOOC).}

\begin{figure}[t]
  \centering
  \begin{minipage}{.5\textwidth}
    \centering
    \includegraphics[width=1\textwidth]{fig_kldiv_koller_mb4_gpu}
    \caption{The $KL_{\rm avg}$ from BIDMach.}
    \label{fig:kl_bidmach}
  \end{minipage}\hfill
    \begin{minipage}{.5\textwidth}
    \centering
    \includegraphics[width=1\textwidth]{fig_kl_div_25_50_perc_jags}
    \caption{The $KL_{\rm avg}$ from JAGS.}
    \label{fig:kl_jags}
  \end{minipage}
\end{figure}

% BIDMach (CPU) vs JAGS for the Koller data. This one only has 50k columns now!
% BIDMach settings: 25k batch size, 200 iterations, 777 seed.  We do get gflops with BIDMach, so we
% can include them if desired, but the runtime is the more interesting thing we want to compare.
% Perhaps we should use total seconds, rather than per iteration? It's easier to understand.
% Also, opts.updateAll = true because we should be updating after each mini-batch.
%
% Here are the BIDMach CPU (not GPU!) results ON BITTER, with extra values if we can get more JAGS results:
%
% Time=8.7060 secs, gflops=0.84 (m=1)
% Time=15.4930 secs, gflops=0.94 (m=2)
% Time=29.7660 secs, gflops=1.23 (m=5)
% Time=67.6580 secs, gflops=1.08 (m=10)
% Time=129.7780 secs, gflops=1.13 (m=20)
%
% BIDMach CPU times on STOUT, which is what we actually have to report!
%
% Time=10.6480 secs, gflops=0.69 (m=1)
% Time=15.3780 secs, gflops=0.95 (m=2)
% Time=31.4540 secs, gflops=1.16 (m=5)
% Time=61.1170 secs, gflops=1.20 (m=10)
% Time=122.2520 secs, gflops=1.20 (m=20)
%
% Yeah, a  bit weird that gflops doesn't increase each time ... also, funny how stout is actually
% faster. That works to our *advantage* here!
%
% EDIT: Argh, we may have to re-run this to get times for FOUR mini-batches since that's what I use
% for Fig 2. Results on stout with mini-batch sizes of 12.5k:
%
% Time=11.0200 secs, gflops=0.66 (m=1)
% Time=18.0350 secs, gflops=0.81 (m=2)
% Time=33.0970 secs, gflops=1.10 (m=5)
% Time=62.2550 secs, gflops=1.17 (m=10)
% Time=115.3720 secs, gflops=1.27 (m=20)
%
% Make sure we mention in the paper that we're deliberately handicapping ourselves. EDIT: wait, wow,
% 12500 can even be faster in some cases? Yeah, CPU runtime is tricky, but these are ballparks anyway.
\begin{table}[t]
\caption{BIDMach (CPU) vs. JAGS Runtime on Koller Data}
\label{tab:bidmach_jags_koller}
\begin{center}
\begin{tabular}{ |c|c|c|c|c|c|c| } 
\hline
                  & $m=1$ & $m=2$ & $m=5$ & $m=10$ & $m=20$ \\
\hline \hline
BIDMach Time/Iter & 0.055 & 0.090 & 0.165 & 0.311  & 0.577 \\ 
JAGS Time/Iter    & & & & & \\
\hline
\end{tabular}
\end{center}
\end{table}

We first use synthetic data generated from a toy Bayesian network to show an example of correctness
and the benefits of SAME. The network consists of five variables: $X_0 = {\rm Intelligence}$, $X_1 =
{\rm Difficulty},$ $X_2 = {\rm SAT}$, $X_3 = {\rm Grade}$, and $X_4 = {\rm Letter}$. The directed
edges are $\mathcal{E} = \{(X_0, X_2), (X_0, X_3), (X_1,X_3), (X_3,X_4)\}$, where $(X_i,X_j)$ means
an arrow points from $X_i$ to $X_j$.  Variable $X_3$ is ternary, and all others are binary. The goal
with this network is to model a student taking a class, considering ability metrics (Intelligence
and SAT score), the class difficulty, and the student's resulting grade, which subsequently affects
the quality of a letter of recommendation. All this is from Chapter 3 of~\citet{Koller2009}, which
also lists the true set of CPTs. Due to the name of the first author, we label this data ``Koller''
data to distinguish it from the MOOC data we use in~(\ref{ssec:mooc_data}).

To generate the data for BIDMach, we listed the variables in a topological ordering and directly
sampled their values, where $X_i$ got sampled according to the values of its parents (if any) based
on the true distributions from~\citet{Koller2009}. For our initial experiments, we generated 50,000
samples, so the data is formatted in a $(5\times 50,000)$-dimensional matrix\footnote{It should be
noted that this is only for the sake of comparison with JAGS. BIDMach can handle millions of
samples, but for JAGS, we limited the number of cases to a reasonable number to facilitate
comparisons.}. Then, we randomly hid 50\% of the elements of the data matrix. The objective is to
use Gibbs sampling to estimate the set of CPTs, and see how close they match the true ones.

There are several metrics to evaluate our sampler. The one we employ in this paper is the average
Kullback-Liebler divergence of all the distributions in the set of CPTs, denoted as $KL_{\rm avg}$.
For two distributions $p(x)$ and $q(x)$, the KL-divergence is $\sum_x p(x) \log(p(x)/q(x))$ where we
iterate over $x$ such that $q(x) > 0$. In the ``student'' data, there are eleven probability
distributions that form the set of CPTs. For instance, $X_4$ ``contributes'' three distributions:
$\Pr(X_4 \mid X_3 = 0), \Pr(X_4 \mid X_3 = 1)$, and $\Pr(X_4 \mid X_3 = 2)$, where $X_3$ (the
parent) is fixed. Considering all the variables means that $KL_{\rm avg} = \frac{1}{11}
\sum_{i=1}^{11} p_i(x) \log(p_i(x)/q_i(x))$ with $q_i$ the distribution our sampler estimates. We do
not work with the full joint distribution $P(X_1,X_2,X_3,X_4,X_5)$ since it is more straightforward
to work with the ``projected'' CPTs, and because with higher dimensional data, computing the full
joint is computationally intractable and almost all entries would have probabilities essentially at
zero.

Figures~\ref{fig:kl_bidmach} and~\ref{fig:kl_jags} plot the $KL_{\rm avg}$ metric for the student
data using BIDMach and JAGS, respectively (note the log scale), with three SAME replication factors.
Our results indicate that the $KL_{\rm avg}$ for BIDMach and JAGS reach roughly the same values,
with a slight advantage to JAGS, though this is in part because of the log scale on the graph. In
practice, the difference between BIDMach and JAGS is indistinguishable to humans, and both versions
are able to get to the true CPTs to within a tolerance of 0.005.  \textbf{Daniel: I need to
formalize this ``tolerance'' and explain what this is like ``in practice'' since that will give more
intuition.  This is in progress.} Both versions converge to the true distributions quickly after
about ten passes. For BIDMach, we used two batches with 25,000 cases each (more batches result in
faster convergence), but initialized the CPTs randomly, which may have slowed down the initial
convergence.

In addition, we observe that increasing $m$ results in CPT estimates that more closely match the
true CPTs. The red curves for Figures~\ref{fig:kl_bidmach} and~\ref{fig:kl_jags} correspond to
substantially worse $KL_{\rm avg}$ than the respective yellow and blue curves. The increase from
$m=1$ to $m=5$ has a much larger relative improvement than the increase from $m=5$ to $m=10$,
consistent with our observations of diminishing returns.

Next, we evaluate how the SAME parameter $m$ affects the runtime of the sampler, since more samples
means the algorithm necessarily runs longer. Figures~\ref{fig:kl_time_bidmach}
and~\ref{fig:kl_time_jags} plot $KL_{\rm avg}$ versus total runtime (in seconds) for BIDMach and
JAGS, respectively. As before, $KL_{\rm avg}$ is on a log scale.

There are several interesting observations. The first is that the curves for higher $m$ values start
later due to high initialization costs (this is especially problematic for JAGS). This, along with
the relative marginal benefit of increasing $m$ on this data means that SAME may not have a
beneficial time-accuracy tradeoff. BIDMach, on the other had, has faster initialization costs and
increasing $m$ results in faster convergence.

% Daniel: I just think we have to get rid of (or replace, more likely) these figures.
%\begin{figure}[t]
%  \centering
%  \begin{minipage}{.5\textwidth}
%    \centering
%    \includegraphics[width=1\textwidth]{fig_kl_time_tradeoff_koller_data}
%    \caption{Time and $KL_{\rm avg}$ (BIDMach).}
%    \label{fig:kl_time_bidmach}
%  \end{minipage}\hfill
%    \begin{minipage}{.5\textwidth}
%    \centering
%    \includegraphics[width=1\textwidth]{fig_kl_div_25_50_perc_jags_time}
%    \caption{Time and $KL_{\rm avg}$ (JAGS).}
%    \label{fig:kl_time_jags}
%  \end{minipage}
%\end{figure}

\subsection{Real Dynamic Learning Maps ``MOOC'' Data}\label{ssec:mooc_data}

% BIDMach (CPU) vs JAGS for the Real MOOC data *replicated*, with SAME=1.
% Batch size: we'll use 2184 (approx 4367/2) to be a fair comparision with previous table with
% Koller, because we usually like to have more than one mini-batch.
%
% Here are the BIDMach CPU (not GPU!) results on BITTER:
%
% Time=36.4890 secs, gflops=1.20 (1x)
% Time=71.5470 secs, gflops=1.22 (2x)
% Time=196.5920 secs, gflops=1.11 (5x)
% Time=387.2260 secs, gflops=1.13 (10x)
% Time=739.8330 secs, gflops=1.18 (20x)
% 
% Here are the CPU BIDMach times on STOUT with the same settings, which is what we report:
%
% Time=38.4290 secs, gflops=1.14 (1x)
% Time=75.0240 secs, gflops=1.17 (2x)
% Time=186.2360 secs, gflops=1.17 (5x)
% Time=358.4780 secs, gflops=1.22 (10x)
% Time=700.2980 secs, gflops=1.25 (20x)
% Time=1436.2930 secs, gflops=1.22 (40x)
%
% It takes a while with 20x, and absurdly long with 100x ... remember, the batch sizes are the same.
% Note: I suppose I should run this with 40% data as well...
%
% Notes on JAGS: we tried with 100x, but it crashed (ran out of memory) on stout, after almost 30
% minutes. And actually, 40% crashed as well. But on stout, BIDMach only uses 11.4% of the memory
% with 40% of the data.
\begin{table}[t]
\caption{BIDMach (CPU) vs. JAGS Runtime on (Replicated) Real MOOC Data}
\label{tab:bidmach_jags_realmooc}
\begin{center}
\begin{tabular}{ |c|c|c|c|c|c|c| } 
\hline
                  & 1x    & 2x    & 5x    & 10x   & 20x   & 40x   \\
\hline \hline
BIDMach Time/Iter & 0.182 & 0.375 & 0.931 & 1.792 & 3.501 & 7.181 \\ 
JAGS Time/Iter    & & & & & & OOM \\
\hline
\end{tabular}
\end{center}
\end{table}

We now benchmark our code on a Bayesian network with a nation-wide examination dataset from the
Dynamic Learning Maps (DLM) project. The data contains the assessment (correct or not) of student
responses to questions from the DLM Alternate Assessment System. After our preprocessing, there were
4367 students and 319 questions. From the DLM data source, each of the questions is considered to be
derived from a set of 15 basic concepts. We encode questions and concepts as variables and describe
their relationships with a Bayesian network from the DLM data. Each question is considered an
``observed'' node in the Bayesian network, but with a very high missing value rate, as students only
got to answer a fraction of the questions.  Our data matrix, which has dimensions $(334\times
4367)$, ultimately ended up with a 2.2\% sparsity level. Furthermore, the first 15 rows in the data
matrix are completely missing across all 4367 columns, since the 15 concepts are latent variables. The
inference task is to learn the CPTs of the Bayesian network from this extremely sparse data.
\textbf{Daniel: I need a reference (maybe we can ask Huasha?) to help me describe this data better.}

We evaluate our sampler on this MOOC data using two methods. The first involves running our sampler
on the original data until convergence. Then, we use the estimated set of CPTs and sample from that
via direct sampling to generate synthetic data, also of dimension $(334 \times 4367)$. We modify
that synthetic data matrix so that it also has a 2.2\% sparsity level and has missing data at the
same elements as the original data matrix. Then we re-run BIDMach on that data and test using the
same $KL_{\rm avg}$ metric from~(\ref{ssec:koller_data}), since we know the CPTs that ``generated''
the data.

Figure~\ref{fig:mooc_kl} plots the $KL_{\rm avg}$ for BIDMach with varying $m$ values. We ran the
sampler for several thousands of iterations but only show the first 300 for the sake of readability.
We see that, indeed, using $m=1$ as a reference, the Gibbs sampler appears to converge quickly to
the true distribution. It seems like the curve gets worse after the first 100 iterations, but
the long term trend (not shown here) is that it settles down at the same $KL_{\rm avg}$ from the 100
iteration point. It is also apparent from Figure~\ref{fig:mooc_kl} (and the long-term trend) that
$m=2$ and $m=5$ are better, showing that there is indeed a benefit to SAME with extremely sparse
data. What is interesting is that the performance of $m=10$ is \emph{worse} than $m=1$. Even after
thousands of iterations, the $m=10$ curve was essentially neck-to-neck with the $m=1$ curve, but
corresponds to a much slower runtime.

\textbf{Daniel: We are planning on including one JAGS curve ($m=1$) in Figure~\ref{fig:mooc_kl}. We
will need to explicitly state the runtimes for BIDMach/JAGS, probably just here in the text and not
in a figure/table. Also, I should explain the ``long term'' trend stuff better, perhaps even
extending the plot a bit. Finally, like before, I want to give an intuitive feel of the accuracy,
i.e., how many decimal points of accuracy?}

\begin{figure}[t]
  \centering
  \begin{minipage}{.5\textwidth}
    \centering
    \includegraphics[width=1\textwidth]{fig_mooc_kl_div.png}
    \caption{MOOC Data $KL_{\rm avg}$. \textbf{Daniel: Don't worry, I will fix the y-axis labels.}}
    \label{fig:mooc_kl}
  \end{minipage}\hfill
    \begin{minipage}{.5\textwidth}
    \centering
    \includegraphics[width=1\textwidth]{fig_prediction_accuracy_mooc}
    \caption{MOOC data prediction accuracy.}
    \label{fig:mooc_accuracy}
  \end{minipage}
\end{figure}

The second way we evaluate our sampler is with prediction accuracy. We divide the original
data into a training and testing batch so that 80\% of the known data is in training. The training
set is for updating the CPTs. During the testing batch, we sample for 500 iterations using the
current CPTs (but \emph{without updating} them). For each of the known test points, we compute the
number of 0s and 1s sampled (from the 500 iterations), pick the majority to be our prediction,
and then compare it with the true value. Since only 2.2\% of the data is known, this is a challenging
prediction problem. Moreover, we divide the training and testing data into batches, so for each
student (i.e., column) in the test data matrix, we have no evidence.  Random guessing is equivalent
to 50\% accuracy since this is a binary prediction task.

We ran the prediction accuracy algorithm with SAME parameters $m=1$ and $m=5$ using three different
random seeds for each, to reduce the chances of one trial skewing the results.
Figure~\ref{fig:mooc_accuracy} plots the mean prediction accuracy (among three trials) for both SAME
parameters. Our sampler indeed learns to predict reasonably well and reaches roughly 61\% accuracy.
SAME is not beneficial in terms of accuracy or even the standard deviation of the accuracies (not
plotted here). This is likely because sampling 500 times to determine our prediction may cause
convergence to a result invariant to minor differences in CPTs.  Sampling only one time to determine
prediction accuracy resulted in about 56\% accuracy for $m=1$, and beyond 500 samples, we did not
observe improvements.

\subsection{Runtime vs. Throughput Tradeoff}\label{ssec:tradeoff}

% This should be in another section. We might consider splitting this in two if we have space. For
% both of these, I chose the largest batch size that we could do m = 50.
%
% Koller Data = (5 x 1,000,000)-dimensional, mini-batch size = 50,000, iterations = 200, seed = 777.
% This is with the sliding, rectangular window of samples, which is *slower* than the moving
% average,  hence we'll get slower gflops. Before, with complete truncation (but an incorrect
% graphical model, technically speaking) we were getting 13.90 for m = 50.
% Final statistics from BIDMach:
% Time=61.4750 secs, gflops=2.38 (m=1)
% Time=104.6140 secs, gflops=6.98 (m=5)
% Time=160.8270 secs, gflops=9.08 (m=10)
% Time=265.6820 secs, gflops=11.00 (m=20)
% Time=374.4060 secs, gflops=11.70 (m=30)
% Time=476.0510 secs, gflops=12.27 (m=40)
% Time=591.2360 secs, gflops=12.35 (m=50)
%
% MOOC Data = (334 x 43670)-dimensional, where we replicated the *real* mooc 5x column-wise. Hence,
% RM = Replicated MOOC. Again, this is with the sliding, rectangular window of samples. I was going
% to use 10x, but I would have had to shrink the batch size down to 100 (or even less than that), I
% think because putback requires extra memory or something. So I'm doing 5x, which lets me increase
% the batch size to 400. More advanced GPUs in the future will allow us to run this faster.

% Settings: mini-batch size = 400, SAME is the variable. 200 iterations and 777 seed as usual.
% Total runtimes were:
%
% Time=133.2460 secs, gflops=1.65 (m=1)
% Time=238.3670 secs, gflops=4.58 (m=5)
% Time=308.1950 secs, gflops=7.08 (m=10)
% Time=450.3450 secs, gflops=9.69 (m=20)
% Time=599.5060 secs, gflops=10.92 (m=30)
% Time=750.6940 secs, gflops=11.62 (m=40)
% Time=907.2790 secs, gflops=12.02 (m=50)
%
% Now for the actual table itself:
\begin{table}[t]
\caption{BIDMach (GPU) Runtime vs. GigaFlops on Large Data}
\label{tab:tradeoff}
\begin{center}
\begin{tabular}{ |c|c|c|c|c|c|c|c| } 
\hline
               & $m=1$ & $m=5$ & $m=10$ & $m=20$ & $m=30$ & $m=40$ & $m=50$  \\
\hline \hline
GigaFlops (K)  & 2.38  & 6.98  & 9.08   & 11.00  & 11.70  & 12.27  & 12.35   \\ 
Time/Iter (K)  & 0.307 & 0.523 & 0.804  & 1.328  & 1.872  & 2.380  & 2.957   \\
\hline 
GigaFlops (RM) & 1.65  & 4.58  & 7.08   & 9.69   & 10.92  & 11.62  & 12.02   \\ 
Time/Iter (RM) & 0.666 & 1.192 & 1.541  & 2.252  & 2.998  & 3.753  & 4.536   \\
\hline
\end{tabular}
\end{center}
\end{table}

We now discuss the impact of the SAME parameter in terms of runtime and throughput. We measure the
throughput and speed of our Gibbs sampler in terms of GigaFlops (gflops), a billion floating point
operations per second. It is possible to compute reasonable estimates of the maximum gflops
attainable for a computer. Therefore, in order for us to claim that the Gibbs sampler is as
optimized as it could be, we should argue that it attains gflops values that are close to the
theoretical limit, indicating that further improvement is unlikely beyond simply upgrading the
hardware.

\textbf{Daniel: We should compute the theoretical gflops limit.}

As the SAME parameter increases, it increases both the throughput (good) and runtime (bad) by
increasing the number of data points in our computations. The results from~\citep{SAME2015} suggest
that increasing $m$ for small values will increase throughput while not costing too much in runtime.
As one increases $m$ beyond a certain data-dependent value, then SAME ``saturates'' the algorithm
and results in stagnant throughput while significantly increasing runtime. It follows that a
reasonable objective for an experimenter is to run the Gibbs sampler on data with varying values of
$m$, and identify the point where further increases in $m$ start having an unfavorable runtime and
throughput tradeoff.

As an example, we used the Koller data from~(\ref{ssec:koller_data}), with 50\% of the data known
versus missing, but with \emph{one million} cases, beyond the feasible datasets of standard Gibbs
samplers. We ran the Gibbs sampler for 200 iterations on this data with a batch size of 50,000.
Table~\ref{tab:tradeoff} shows the time and gflops of the Gibbs sampler using different values of
$m$.  One can see that increasing $m$ for $m < 30$ results in steady gflops increases but
\emph{without} a substantial increase in time. Then SAME saturates the algorithm and the runtime
increases while the gflops stalls, reinforcing the conclusions from~\citep{SAME2015} and argues for
the importance of testing with a variety of $m$ values.

Notice that the runtimes listed are in seconds, and even with $m=50$, 200 iterations takes less than
10 minutes to complete --- we have barely scratched the limit of our sampler. In fact, one of the
main things that limits our sampler is the memory of GPUs, since we need to shrink the batch size
with large $m$. As memory on GPUs becomes cheaper, we will be able to run Gibbs sampling and perform
graphical model inference on larger datasets.


\section{Conclusions}\label{sec:conclusions}

We conclude that our Gibbs sampler is much faster than the state of the art (JAGS) in Gibbs sampling
and can be applied to data with several hundreds of variables. We also argue that SAME is beneficial
for Gibbs sampling, and that it should be the go-to method for researchers who wish to perform
inference on (discrete) Bayesian networks. Future work will explore the application of our sampler
to a wider class of real-world datasets.


\subsubsection*{Acknowledgments}

We thank Yang Gao, Biye Jiang, and Huasha Zhao for helpful discussions.

\bibliography{iclr2016_conference}
\bibliographystyle{iclr2016_conference}









%%% APPENDIX
%\clearpage
%\appendix
%
%\textbf{I expect that we will use eight pages for text, plus the ninth page for references. Then the
%remaining material (if any) will go here.}




% More stuff from Huasha:
%\subsection{Performance and Runtime}
%We first look at the efficiency of each system measured by giga floating point operations per second
%(Gflops). Intel VTune Amplifier is used to measure the flops numbers. As presented in Figure
%\ref{perf}, BIDMach achieves 5 Gflops and 1 Gflops for GPU and CPU respectively. The Gibbs sampler
%is bottlenecked by the calculation of sampling probability vectors which is implemented using SpMV
%operations. Such flops numbers are the hardware limit of SpMV operation. Jags and Infer.net operates
%at much lower flops rates. Note that the y-axis of the figure is in log-scale. The VTune profile
%results also show that Jags spend 70 \% of the runtime on disk IO, which is highly inefficient. We
%also observe that the memory usage of Infer.net is not efficient: on our PC with 8G memory, it
%cannot scale up to 10000 students (with the same statistics as the DLM pilot dataset we use).

\begin{comment}
% Daniel: I'm leaving this here for now, because this is some extra stuff from Huasha's old write-up
% that we might use.

We benchmark all the systems on fitting a Bayesian network with a nation-wide examination dataset
from the Dynamic Learning Maps (DLM) project. The dataset contains the assessment (correct or not)
of 30,000 students' responses to questions from the DLM Alternate Assessment System. There are 4000
students and 340 unique questions in the pilot experiment ,and the overall completion rate of the
questions is only 2.2 \% (assessment questions are tailored for each student). Each of the 340
questions is considered to be derived from a set of 15 basic concepts, and relations between
questions and concepts and within concepts are given. Each question is considered as a observed node
in the Bayesian network (with very high missing value rate), and each concept is considered as a
hidden node which never gets observed. Each node takes a binary value. The inference task is to
learn the parameter of the network on 80\% of the response assessment and predict on the rest 20\%
of the response. We use the prediction accuracy to measure the quality of the model. 

\subsection{Performance and Runtime}
We first look at the efficiency of each system measured by giga floating point operations per second
(Gflops). Intel VTune Amplifier is used to measure the flops numbers. As presented in Figure
\ref{perf}, BIDMach achieves 5 Gflops and 1 Gflops for GPU and CPU respectively. The Gibbs sampler
is bottlenecked by the calculation of sampling probability vectors which is implemented using SpMV
operations. Such flops numbers are the hardware limit of SpMV operation. Jags and Infer.net operates
at much lower flops rates. Note that the y-axis of the figure is in log-scale. The VTune profile
results also show that Jags spend 70 \% of the runtime on disk IO, which is highly inefficient. We
also observe that the memory usage of Infer.net is not efficient: on our PC with 8G memory, it
cannot scale up to 10000 students (with the same statistics as the DLM pilot dataset we use).

\begin{figure}[h!]
\centering
\includegraphics[scale = 0.7]{perf2.png}
\caption{Performance Comparison} 
\label{perf}
\end{figure}

Figure \ref{runtime2} shows the runtime until convergence for each inference engine. Again, time is
in log-scale. The Gibbs sample approach converges in about 200 iterations, while the EP algorithm
converges in 50 iterations. Infer.net is 3.5x faster than Jags. This is expected as symbolic method
is usually more efficient than sampling approach. BIDMach is 2-3 orders of magnitude faster than the
other systems. 

We can also verify that BIDMach is doing the same amount of work (floating point operations) as Jags
by multiplying the gflops number in Figure \ref{perf} with the run time in Figure \ref{runtime2}.

\begin{figure}[h!]
\centering
\includegraphics[scale = 0.7]{time2.png}
\caption{Runtime Comparison} 
\label{runtime2}
\end{figure}

\subsection{Prediction Accuracy}
For each node to be predicted, we sample 50 instances for that node from the learned network and
observed values, and then take the majority as the predicted value. The accuracy is measured as the
percentage of corrected predictions. A random guess will give an accuracy of 50\%. Figure
\ref{accuracy} shows the predicted accuracy as a function of number of iterations in training. Both
BIDMach and Jags achieves 65-67\% accuracy in around 200 iterations. However, BIDMach has a huge
advantage in terms of speed as shown in Figure \ref{runtime2}.

\begin{figure}[h!]
\centering
\includegraphics[scale = 0.7]{accuracy.png}
\caption{Accuracy Comparison} 
\label{accuracy}
\end{figure}

\end{comment}




\end{document}


%\section{Citations, figures, tables, references}
%\label{others}
%
%These instructions apply to everyone, regardless of the formatter being used.
%
%\subsection{Citations within the text}
%
%Citations within the text should be based on the {\tt natbib} package and include the authors' last
%names and year (with the ``et~al.'' construct for more than two authors). When the authors or the
%publication are included in the sentence, the citation should not be in parenthesis (as in ``See
%\citet{Hinton06} for more information.''). Otherwise, the citation should be in parenthesis (as in
%``Deep learning shows promise to make progress towards AI~\citep{Bengio+chapter2007}.'').
%
%The corresponding references are to be listed in alphabetical order of authors, in the
%\textsc{References} section. As to the format of the references themselves, any style is acceptable
%as long as it is used consistently.
%
%\subsection{Figures}
%
%All artwork must be neat, clean, and legible. Lines should be dark enough for purposes of
%reproduction; art work should not be hand-drawn. The figure number and caption always appear after
%the figure. Place one line space before the figure caption, and one line space after the figure. The
%figure caption is lower case (except for first word and proper nouns); figures are numbered
%consecutively.
%
%Make sure the figure caption does not get separated from the figure.  Leave sufficient space to
%avoid splitting the figure and figure caption.
%
%You may use color figures.  However, it is best for the figure captions and the paper body to make
%sense if the paper is printed either in black/white or in color.
%\begin{figure}[h]
%\begin{center}
%%\framebox[4.0in]{$\;$}
%\fbox{\rule[-.5cm]{0cm}{4cm} \rule[-.5cm]{4cm}{0cm}}
%\end{center}
%\caption{Sample figure caption.}
%\end{figure}
%
%\subsection{Tables}
%
%All tables must be centered, neat, clean and legible. Do not use hand-drawn tables. The table number
%and title always appear before the table. See Table~\ref{sample-table}.
%
%Place one line space before the table title, one line space after the table title, and one line
%space after the table. The table title must be lower case (except for first word and proper nouns);
%tables are numbered consecutively.
%
%\begin{table}[t]
%\caption{Sample table title}
%\label{sample-table}
%\begin{center}
%\begin{tabular}{ll}
%\multicolumn{1}{c}{\bf PART}  &\multicolumn{1}{c}{\bf DESCRIPTION}
%\\ \hline \\
%Dendrite         &Input terminal \\
%Axon             &Output terminal \\
%Soma             &Cell body (contains cell nucleus) \\
%\end{tabular}
%\end{center}
%\end{table}
%
%\section{Final instructions}
%Do not change any aspects of the formatting parameters in the style files.  In particular, do not
%modify the width or length of the rectangle the text should fit into, and do not change font sizes
%(except perhaps in the \textsc{References} section; see below). Please note that pages should be
%numbered.
%
%\section{Preparing PostScript or PDF files}
%
%Please prepare PostScript or PDF files with paper size ``US Letter'', and not, for example, ``A4''.
%The -t letter option on dvips will produce US Letter files.
%
%Consider directly generating PDF files using \verb+pdflatex+ (especially if you are a MiKTeX user).
%PDF figures must be substituted for EPS figures, however.
%
%Otherwise, please generate your PostScript and PDF files with the following commands:
%\begin{verbatim}
%dvips mypaper.dvi -t letter -Ppdf -G0 -o mypaper.ps
%ps2pdf mypaper.ps mypaper.pdf
%\end{verbatim}
%
%\subsection{Margins in LaTeX}
%
%Most of the margin problems come from figures positioned by hand using \verb+\special+ or other
%commands. We suggest using the command \verb+\includegraphics+ from the graphicx package. Always
%specify the figure width as a multiple of the line width as in the example below using .eps graphics
%\begin{verbatim}
%   \usepackage[dvips]{graphicx} ...
%   \includegraphics[width=0.8\linewidth]{myfile.eps}
%\end{verbatim}
%or % Apr 2009 addition
%\begin{verbatim}
%   \usepackage[pdftex]{graphicx} ...
%   \includegraphics[width=0.8\linewidth]{myfile.pdf}
%\end{verbatim}
%for .pdf graphics.  See section 4.4 in the graphics bundle documentation
%    (\url{http://www.ctan.org/tex-archive/macros/latex/required/graphics/grfguide.ps})
%
%A number of width problems arise when LaTeX cannot properly hyphenate a line. Please give LaTeX
%hyphenation hints using the \verb+\-+ command.
%
%\end{document}
